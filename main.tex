\documentclass[11pt,a4paper]{amsart}

\usepackage{hieroglf,times,amsmath,stmaryrd,tipa}

\usepackage{xcolor, soul,graphicx}
\definecolor{verylightgray}{gray}{0.8}
\sethlcolor{verylightgray}

\usepackage[normalem]{ulem}

\newcommand{\aeg}[2][\normalsize]{{#1\pmglyph{#2}}}
% \def\aeg#1{{\tiny\pmglyph{#1}}}
\def\tran#1{\translitpmhgfont{#1}}
\usepackage{amsmath}
\author{Alessandro Roccati}
\title{Elementi di Lingua Egizia}
\date{\today}

\begin{document}
\maketitle
\section*{Prefazione}
Questi ``elementi'' costituiscono un semplice schizzo grammaticale e non vogliono sostituire in alcun modo l'uso dei manuali, soprattutto quello del Gardiner \cite{gardiner1957egyptian}. Essi li possono però affiancare, poiché la loro redazione è dovuta non solo al desiderio di provvedere un sussidio didattico in italiano.
\section{Introduzione}
\section{La scrittura geroglifica}
\subsection{Cenni sul sistema figurativo}
\subsection{Il sistema fonetico}
[...]
Altri segni specificano gruppi di due o di tre consonanti, con riferimento ai suoni sopra indicati. La tabella seguente indica insieme i valori e le possibilità di combinazione, es. \aeg[\tiny]{\HC:\Hn-\Hi-\Hm-\Hms-\Hs}
[...]
% \documentclass{standalone}
\usepackage{hieroglf,times,amsmath,stmaryrd,tipa}

\usepackage{xcolor, soul}
\definecolor{verylightgray}{gray}{0.8}
\sethlcolor{verylightgray}

\usepackage[normalem]{ulem}

\newcommand{\aeg}[2][\normalsize]{{#1\pmglyph{#2}}}
% \def\aeg#1{{\tiny\pmglyph{#1}}}
\def\tran#1{\translitpmhgfont{#1}}
\usepackage{amsmath}

\begin{document}
\def\sp{\kern.25em}
% \begin{table}[h!]
	\begin{tabular}{|c||ccccccccccccccccccccc}\hline
		C+                        & \textyogh        & \tran{\Hic} & `   & w                 & b              & p         & m           & n                            & r         & \d{h} & z           & s              & \d{k} & k & t   & d   & \underline{d} \\ \hline\hline
		\textyogh\sp\aeg{\Ha}     &                  &             &     & \aeg{\HZ}         & ???            &           &             &                              &           &       &             &                &       &   &     &     &               \\\hline
		\tran{\Hic}\sp\aeg{\Hi}   &                  &             &     & \aeg{\HL}         & \aeg{\HFxxxiv} &           & \aeg{\HM}   & \aeg{\HF}                    & \aeg{\He} &       & \aeg{\HYiV} &                &       &   &     & ??? &               \\\hline
		`\sp\aeg{\HA}             & ???              &             &     &                   &                &           &             &                              &           &       &             &                & ???   &   &     &     & ???           \\\hline
		w\sp\aeg{\Hw}             & \aeg{\Ho}        &             & ??? &                   &                & ???       &             & ???                          & \aeg{\HR} &       &             &                &       &   &     &     & ???           \\\hline
		b\sp\aeg{\Hb}             & \aeg{\Hibw}      &             &     &                   &                &           &             &                              &           & ???   &             &                &       &   &     &     &               \\\hline
		p\sp\aeg{\Hp}             & ???              &             &     &                   &                &           &             &                              & \aeg{\Hj} & ???   &             &                &       &   &     &     &               \\\hline
		m\sp\aeg{\Hm}             & ???              & ???         &     & \aeg{\Hn:\Hn:\Hn} &                &           &             & ???                          & ???       & ???   &             & ???            &       &   & ??? &     & ???           \\\hline
		n\sp\aeg{\Hn}             &                  & ???         &     & ???               & ???            &           & ???         & \aeg{\HSxxxix}\aeg{\HSxxxix} & ???       & ???   &             & \aeg{\Htongue} &       &   &     &     & ???           \\\hline
		r\sp\aeg{\Hr}             &                  &             &     & ???               &                &           &             &                              &           &       &             & ???            &       &   &     &     &               \\\hline
		h\sp\aeg{\HH}             & ???              &             &     & ???               &                & ???       & \aeg{\HJ}   & ???                          & \aeg{\Hq} &       & ???         &                &       &   &     &     & ???           \\\hline
		\uunder{h}\sp\aeg{\HC}    & \aeg{\Hthousand} &             & ??? & ???               &                &           &             &                              & ???       &       &             &                &       &   & ??? &     &               \\\hline
		\underline{h}\sp???       & ???              &             &     &                   &                &           &             & ???                          & ???       &       &             &                &       &   &     &     &               \\\hline
		z\sp\aeg{\HS}             & ???              &             &     &                   &                &           &             &                              &           &       &             &                &       &   &     &     &               \\\hline
		s\sp\aeg{\Hs}             & ???              &             &     & ???               &                &           &             & ???                          &           &       &             &                &       & ? & ??? &     &               \\\hline
		\v{s}\sp\aeg{\Hz}         & \aeg{\HE}        &             &     & ???               &                &           &             & ???                          &           &       &             & ???            &       &   &     & ??? &               \\\hline
		\d{k}\sp\aeg{\HK}         &                  &             &     &                   & ???            &           &             &                              &           &       &             &                &       &   &     & ??? &               \\\hline
		k\sp\aeg{\Hk}             & ???              &             &     &                   &                & ???       & ???         &                              &           &       &             &                &       &   &     &     &               \\\hline
		g\sp\aeg{\Hg}             & ???              &             &     &                   &                &           & \aeg{\Hibl} &                              &           &       &             & \aeg{\HAaxii}  &       &   &     &     &               \\\hline
		t\sp\aeg{\Ht}             & ???              & ???         &     & ???               &                & \aeg{\HQ} & ???         &                              & ???       &       &             &                &       &   &     &     &               \\\hline
		\underline{t}\sp\aeg{\HT} & ???              &             &     &                   &                &           &             &                              &           &       & ???         &                &       &   &     &     &               \\\hline
		d\sp\aeg{\Hd}             &                  &             &     &                   &                &           &             &                              &           &       &             &                &       &   &     &     &               \\\hline
		\underline{d}\sp\aeg{\HD} & ???              &             &     & ???               &                &           & \aeg{\Hc}   & ???                          & ???       &       &             &                &       &   &     & ??? &               \\\hline
	\end{tabular}
% 	\caption{Prospetto dei segni dilitteri più comuni}
% \end{table}
\end{document}
\begin{table}[h!]
 \includegraphics[width=\textwidth]{table1}
\end{table}
\subsection{Il sistema ausiliario}
\section{Numeri}
\section{Fonetica}
\section{Pronomi Personali}
\subsection{I. Pronomi suffissi}
\subsection{II. Pronomi dipendenti o enclitici}
\subsection{III. Pronomi indipendenti o proclitici}
\subsection{IV. Formazione analitica del possessivo}
\subsection{Espressione dell'agente}
\section{Dimostrativi}
\section{Nome}
\subsection{Attributi}
\section{Sintassi del nome}
\subsection{Accostamento di nomi}
\subsection{Preposizioni}
\subsection{Nomi di relazione}
\subsection{La frase nominale}
\section{Verbo}
\subsection{Forme del tema}
\subsection{Coniugazione}
\subsection{Coniugazione fondamentale}
\subsection{Forme finite semplici}
\subsection{Forma finita ampliata}
\subsubsection{Uso non attributivo}
\subsubsection{Uso attributivo}
\subsection{Temi ampliati}
\subsection{Imperativo}
\subsection{Pseudoparticipio}
\subsection{Forme composte}
\subsection{Scambio di Forme}
\section{Sintassi dell'azione}
\subsection{Soggetto e predicato}
\subsection{Frase esprimente azione}
\subsection{Accostamento di azioni}
\subsection{La frase negativa}
\section{Sintassi delle frasi}
\section{Espressioni lessicali}

\nocite{*}
\bibliography{biblio.bib}
\bibliographystyle{alpha}
\end{document}